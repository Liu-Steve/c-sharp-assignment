\documentclass{article}
\usepackage{ctex}
\usepackage{float}
\usepackage{geometry}
\usepackage{graphicx}
\usepackage{diagbox}
\geometry{a4paper,scale=0.8}
\title{C\#大作业开题报告}
\author{谭骞 \quad 刘睿尧 \quad 曹雨欣}
\date{}
\begin{document}
\maketitle
\section*{选题动机}
我国公共交通行业蓬勃发展,拥有世界上规模最大的城市公共交通网络, 公交市场规模庞大。然而,近年来公交车突发安全事故屡见不鲜,如 10.28 重庆公交坠江事故、7.7 安顺公交车坠湖事故等,造成人员伤亡和重大财产损 失。 公交车突发安全事故主要集中于以下几种情况: 

(1)司机突发心梗、脑梗、脑出血、猝死等,失去意识及控制能力。 

(2)司机违规驾驶、疲劳驾驶,酒驾、毒驾、开车打电话、开小差、单手开车等。 

(3)乘客打骂甚至劫持司机,干扰到司机的操作。 

基此,本项目提出一款保障公交车驾驶安全,关注公交司机身体健康的安全驾驶监测系统。并将其中的云端部分作为C\#结课大作业提交。
\section*{软件功能}
\subsection*{管理员云平台}
\begin{enumerate}
\item 管理员打开后,先登录,并选择所在的地区
\item 登录之后显示的是地图界面,展示所有公交车线路。没有异常状况时每一条线路上标绿
\item 管理员可以点击任一条线路,该线路可以放大,只显示在这一条线路运行的公交车,用汽车图标标出这条线路上全部公交车的位置
\item 再点击某一辆车,则可以查看当前车内的监控画面、司机的各项指标、车辆当前位置和下一站
\item 出现异常情况或者司机的主动呼救时,界面弹出异常提示信息,信息包括当前状况的分析,以及司机电话号码、当前的位置,并且伴随提示音
\item 关闭弹窗后,就会回到这个司机的详情界面,如果出现多个司机出问题就分屏变多
\item 管理员可以和司机沟通交流,提供实时通话按钮和留言按钮
\item 管理员有信息箱,司机的留言会存在信息箱中
\end{enumerate}
\subsection*{asp.net后端}
\begin{enumerate}
\item 我们会训练一个CNN部署在后端,用于对监控视频中司机状态进行监控分析
\item 完成与前端页面的交互功能,能及时更新页面上的数据
\item 能传输视频流,并显示到页面上
\item 将获取到的数据存入数据库内
\end{enumerate}
\section*{技术路线}
\begin{enumerate}
\item JS+vue进行前端页面编写
\item asp.net框架进行后端编写
\item MySQL作为数据库,进行数据的存储
\item 使用PyTorch库编写模型
\item HTTP传输视频流
\end{enumerate}
\section*{难点}
\begin{enumerate}
\item 并发及异步编程
\item 数据库信息存储
\item 视频流的传输
\item CNN模型的训练
\item 流畅美观的前端交互页面
\end{enumerate}
\section*{成员分工}
\begin{table}[H]
\begin{center}
\caption{成员分工}
\begin{tabular}{|c|c|c|c|}
\hline
\diagbox{任务}{人员}&谭骞&刘睿尧&曹雨欣\\
\hline
前端页面编写&√&√&\\
\hline
后端逻辑交互代码&√&&√\\
\hline
数据库&&&√\\
\hline
模型训练&√&√&√\\
\hline
视频流传输&&√&\\
\hline
\end{tabular}
\end{center}
\end{table}
\end{document}